\documentclass[12pt]{article}
\usepackage{xeCJK}
\usepackage{fontspec}
\usepackage[a4paper,top=2.8cm,bottom=2.8cm,left=2.3cm,right=2.3cm]{geometry}
\usepackage{graphicx}
\usepackage{listings}
\usepackage{xcolor}
\usepackage{indentfirst}
\usepackage{tikz}
\usepackage{amssymb}
\usepackage{amsthm}
\usepackage{amsmath}
\usepackage{fancyhdr}
\usepackage{tabularx}
\usepackage{hyperref}
\usepackage{ulem}
\usepackage{version}
\usepackage{thmtools}
\usepackage{qtree}
\usepackage{algpseudocode}
\usepackage{mathtools}
\usepackage{multicol}
\usepackage{enumitem}
\usepackage{ctable}
\usepackage{totcount}
\usepackage{textpos}

\XeTeXlinebreaklocale "zh"
\XeTeXlinebreakskip = 0pt plus 1pt

\setCJKmainfont[BoldFont={SourceHanSerifTC-SemiBold.otf},AutoFakeSlant]{SourceHanSerifTC-Regular.otf}
\setmonofont{CascadiaCode-Regular.otf}
\newfontfamily{\ProblemTitleMainFont}{SourceHanSerifTC-Bold.otf}
\newCJKfontfamily{\ProblemTitleCJKFont}{SourceHanSerifTC-Bold.otf}
\newcommand{\ProblemTitleFont}{\ProblemTitleMainFont\ProblemTitleCJKFont}

\pagestyle{fancy}

\lstset{
basicstyle=\footnotesize\ttfamily
}

% \raggedcolumns

\newcommand{\ProblemTitle}[2]{\noindent\Large{\ProblemTitleFont #1 (#2)}\normalsize\par}
\newcommand{\ProblemSection}[1]{\vspace{0.6cm}\par\noindent\large{\ProblemTitleFont #1}\normalsize\par}
\newcommand{\ProblemSubsection}[1]{\par\noindent{\ProblemTitleFont #1}\par}
\newcommand{\ProblemStatement}{\ProblemSection{問題敘述}}
\newcommand{\ProblemInput}{\ProblemSection{輸入說明}}
\newcommand{\ProblemOutput}{\ProblemSection{輸出說明}}
\newcommand{\ProblemConstraints}{\ProblemSection{測資限制}}

\newcommand{\ProblemSampleTitle}{\ProblemSection{範例測資}}

\newcounter{ProblemSample}
\newcommand{\ProblemSample}[2]{
    \stepcounter{ProblemSample}
    \noindent
    \begin{minipage}[t]{0.5\textwidth}
        \ProblemSubsection{範例輸入 \theProblemSample}
        \lstinputlisting{#1}
    \end{minipage}
    \begin{minipage}[t]{0.5\textwidth}
        \ProblemSubsection{範例輸出 \theProblemSample}
        \lstinputlisting{#2}
    \end{minipage}
}
\newenvironment{ProblemSampleWithNote}[2]{
    \stepcounter{ProblemSample}
    \noindent
    \begin{minipage}[t]{0.5\textwidth}
        \ProblemSubsection{範例輸入 \theProblemSample}
        \lstinputlisting{#1}
    \end{minipage}
    \begin{minipage}[t]{0.5\textwidth}
        \ProblemSubsection{範例輸出 \theProblemSample}
        \lstinputlisting{#2}
    \end{minipage}
    \vspace{-0.4cm}
    \ProblemSubsection{範例說明 \theProblemSample}
}{}

\newcommand{\ProblemSubtaskTitle}{\ProblemSection{評分說明}}
\newtotcounter{ProblemSubtask}
\newenvironment{ProblemSubtaskTable}{
    \begin{center}
        \begin{tabular}{ccl}
            \specialrule{1.3pt}{0pt}{1pt}
            子任務 & 分數 & 額外輸入限制 \\
            \specialrule{0.5pt}{1pt}{1pt}
}
{
            \specialrule{1.3pt}{1pt}{0pt}
        \end{tabular}
    \end{center}
}
\newcommand{\ProblemSubtask}[2]{ \stepcounter{ProblemSubtask} \theProblemSubtask & #1 & #2 \\ }

\setlist[enumerate]{itemsep=0pt, parsep=0pt, topsep=0pt}
\setlist[itemize]{itemsep=0pt, parsep=0pt, topsep=0pt}

\begin{document}



\renewcommand{\headrulewidth}{0pt}
\renewcommand{\baselinestretch}{1.3}
\pagenumbering{arabic}
\setlength\parindent{24pt}
\setlength\parskip{12pt}
\cfoot{\thepage}
\rhead{
	\small{SCIST S3 演算法期中考}
	
}

\ProblemTitle{C. 神奇的環狀數列}{Circle}

\ProblemStatement

給定一個神奇的環狀數列(代表這個數列的最後一項是跟第一項相連接的),裡面有 $n$ 個數。每個數有一個值 $a_i$ 代表經過此格可以獲得的分數。

這個環狀數列有一個特性是,每一格只要被經過一次這格的 $a_i$ 就會減 $1$。
(如果 $a_i$ 為 $0$,那麼 $a_i$ 便不會再 $-1$)

你可以從環狀數列的任意一個地方開始,固定朝著一個方向走 $k$ 步(開始的地方有算 $1$ 步)。

請問,最高可以得到的分數是多少??

\ProblemInput

第一行有兩個數字 $n,k$ 。分別代表數列的長度以及最大可以走的步數。

第二行有 $n$ 個正整數 $a_1,a_2,a_3, \dots , a_n$ 代表經過第 $i$ 個格子可以拿到的分數。

\ProblemOutput

請輸出一個數字代表最高可以獲得的點數。

%\clearpage

\ProblemConstraints

\begin{itemize}
    \item $2 \leq n \leq 2 \times 10^5$
    \item $1 \leq k \leq 10^{9}$ 
    \item $1 \leq a_i \leq 5 \times 10^{4}$ 
\end{itemize}

\clearpage

\ProblemSampleTitle

\begin{ProblemSampleWithNote}{1in.txt}{1out.txt}
    範例一中,從任意一個地方開始走 $5$ 步皆可以拿到最高分 $15$ 分。
\end{ProblemSampleWithNote}

\begin{ProblemSampleWithNote}{2in.txt}{2out.txt}
    範例二中,從第四個開始走,可以拿到 $4+5+1+2+3+3+4$ 分。
\end{ProblemSampleWithNote}


% \ProblemSample{1in.txt}{1out.txt}
% \ProblemSample{2in.txt}{2out.txt}
\ProblemSample{3in.txt}{3out.txt}


\ProblemSubtaskTitle

本題共有 3 組子任務,條件限制如下所示。

\begin{ProblemSubtaskTable}
    \ProblemSubtask{30}{$n \leq 10^3 , k \leq 10^4$}
    \ProblemSubtask{20}{$n \le 10^4$}
    \ProblemSubtask{50}{無額外限制}
\end{ProblemSubtaskTable}

\end{document}