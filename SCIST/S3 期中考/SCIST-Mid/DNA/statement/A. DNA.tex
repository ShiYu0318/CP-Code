\documentclass[12pt]{article}
\usepackage{xeCJK}
\usepackage{fontspec}
\usepackage[a4paper,top=2.8cm,bottom=2.8cm,left=2.3cm,right=2.3cm]{geometry}
\usepackage{graphicx}
\usepackage{listings}
\usepackage{xcolor}
\usepackage{indentfirst}
\usepackage{tikz}
\usepackage{amssymb}
\usepackage{amsthm}
\usepackage{amsmath}
\usepackage{fancyhdr}
\usepackage{tabularx}
\usepackage{hyperref}
\usepackage{ulem}
\usepackage{version}
\usepackage{thmtools}
\usepackage{qtree}
\usepackage{algpseudocode}
\usepackage{mathtools}
\usepackage{multicol}
\usepackage{enumitem}
\usepackage{ctable}
\usepackage{totcount}
\usepackage{textpos}

\XeTeXlinebreaklocale "zh"
\XeTeXlinebreakskip = 0pt plus 1pt

\setCJKmainfont[BoldFont={SourceHanSerifTC-SemiBold.otf},AutoFakeSlant]{SourceHanSerifTC-Regular.otf}
\setmonofont{CascadiaCode-Regular.otf}
\newfontfamily{\ProblemTitleMainFont}{SourceHanSerifTC-Bold.otf}
\newCJKfontfamily{\ProblemTitleCJKFont}{SourceHanSerifTC-Bold.otf}
\newcommand{\ProblemTitleFont}{\ProblemTitleMainFont\ProblemTitleCJKFont}

\pagestyle{fancy}

\lstset{
basicstyle=\footnotesize\ttfamily
}

% \raggedcolumns

\newcommand{\ProblemTitle}[2]{\noindent\Large{\ProblemTitleFont #1 (#2)}\normalsize\par}
\newcommand{\ProblemSection}[1]{\vspace{0.6cm}\par\noindent\large{\ProblemTitleFont #1}\normalsize\par}
\newcommand{\ProblemSubsection}[1]{\par\noindent{\ProblemTitleFont #1}\par}
\newcommand{\ProblemStatement}{\ProblemSection{問題敘述}}
\newcommand{\ProblemInput}{\ProblemSection{輸入說明}}
\newcommand{\ProblemOutput}{\ProblemSection{輸出說明}}
\newcommand{\ProblemConstraints}{\ProblemSection{測資限制}}

\newcommand{\ProblemSampleTitle}{\ProblemSection{範例測資}}

\newcounter{ProblemSample}
\newcommand{\ProblemSample}[2]{
    \stepcounter{ProblemSample}
    \noindent
    \begin{minipage}[t]{0.5\textwidth}
        \ProblemSubsection{範例輸入 \theProblemSample}
        \lstinputlisting{#1}
    \end{minipage}
    \begin{minipage}[t]{0.5\textwidth}
        \ProblemSubsection{範例輸出 \theProblemSample}
        \lstinputlisting{#2}
    \end{minipage}
}
\newenvironment{ProblemSampleWithNote}[2]{
    \stepcounter{ProblemSample}
    \noindent
    \begin{minipage}[t]{0.5\textwidth}
        \ProblemSubsection{範例輸入 \theProblemSample}
        \lstinputlisting{#1}
    \end{minipage}
    \begin{minipage}[t]{0.5\textwidth}
        \ProblemSubsection{範例輸出 \theProblemSample}
        \lstinputlisting{#2}
    \end{minipage}
    \vspace{-0.4cm}
    \ProblemSubsection{範例說明 \theProblemSample}
}{}

\newcommand{\ProblemSubtaskTitle}{\ProblemSection{評分說明}}
\newtotcounter{ProblemSubtask}
\newenvironment{ProblemSubtaskTable}{
    \begin{center}
        \begin{tabular}{ccl}
            \specialrule{1.3pt}{0pt}{1pt}
            子任務 & 分數 & 額外輸入限制 \\
            \specialrule{0.5pt}{1pt}{1pt}
}
{
            \specialrule{1.3pt}{1pt}{0pt}
        \end{tabular}
    \end{center}
}
\newcommand{\ProblemSubtask}[2]{ \stepcounter{ProblemSubtask} \theProblemSubtask & #1 & #2 \\ }

\setlist[enumerate]{itemsep=0pt, parsep=0pt, topsep=0pt}
\setlist[itemize]{itemsep=0pt, parsep=0pt, topsep=0pt}

\begin{document}



\renewcommand{\headrulewidth}{0pt}
\renewcommand{\baselinestretch}{1.3}
\pagenumbering{arabic}
\setlength\parindent{24pt}
\setlength\parskip{12pt}
\cfoot{\thepage}
\rhead{
	\small{SCIST S3 演算法期中考}
	
}

\ProblemTitle{A. DNA 比對}{DNA}

\ProblemStatement

Koying 是一位研究基因的科學家。他最近在研究電神的 DNA,發現許多電神跟自己都有一個共通點-就是DNA會有一部分是完全相同的。

Fish 知道了這個祕密後,偷偷拿到了 Koying 的基因來進行比對,想要看一下自己與 Koying 電神有沒有共同的地方。

但基因序列實在是太長了,某些 DNA 甚至有高達 $10^5$ 個鹼基。他知道如果要一個一個比對的話會非常困難,於是他想到了一個新的方法。

他決定先將兩個基因序列分成許多小片段,然後比對每一個小片段是否完全相同。他發現這樣做可以大大減少比對的次數,並且更容易找到相似之處。

奈何他實在是太笨了,不會寫程式。所以他想請你幫他寫一個程式,判斷基因序列中的某一個片段是否完全相同。

每次詢問包含兩個數字 $l, r$ 。代表查詢兩個基因的 $[l, r]$ 區間是否完全相同。也就是說 $s_{1_i} = s_{2_i}$ $(l \leq i \leq r)$。

如果查詢後區間完全相同,請輸出"Really?!",否則輸出 "I am so weak."。(不含雙引號)

\ProblemInput

第 $1, 2$ 行分別有兩個長度相同的字串$s_1, s_2$,代表兩人的基因序列。

第三行有一個數字 $q$ 代表詢問的數量 

接下來有 $q$ 行 每行有兩個數字 $l, r$ 代表要查詢的區間。

\ProblemOutput

若查詢後區間完全相同,請輸出"Really?!",否則輸出 "I am so weak."。(不含雙引號)

\clearpage

\ProblemConstraints

\begin{itemize}
    \item $1 \leq |s_1| = |s_2| \leq 10^5$。
    \item $|s_1|,|s_2|$ 分別指的是字串 $s_1,s_2$ 的長度。
    \item $s_1,s_2$ 皆由大寫字母 $A,T,C,G$ 組成。
    \item $1 \leq q \leq 2 \times 10^5$。
    \item $1 \leq l \leq r \leq n$。
\end{itemize}

\ProblemSampleTitle

%\begin{ProblemSampleWithNote}{ex1.in}{ex1.out}

%\end{ProblemSampleWithNote}

\ProblemSample{1in.txt}{1out.txt}
\ProblemSample{2in.txt}{2out.txt}
\ProblemSample{3in.txt}{3out.txt}


\ProblemSubtaskTitle

本題共有 3 組子任務,條件限制如下所示。

\begin{ProblemSubtaskTable}
    \ProblemSubtask{100}{$1 \leq q \leq 100$}
    \ProblemSubtask{0}{$1 \leq |s_1| = |s_2| \leq 2 \times 10^4$}
    \ProblemSubtask{0}{無額外限制}
\end{ProblemSubtaskTable}

\end{document}