\documentclass[12pt]{article}
\usepackage{xeCJK}
\usepackage{fontspec}
\usepackage[a4paper,top=2.8cm,bottom=2.8cm,left=2.3cm,right=2.3cm]{geometry}
\usepackage{graphicx}
\usepackage{listings}
\usepackage{xcolor}
\usepackage{indentfirst}
\usepackage{tikz}
\usepackage{amssymb}
\usepackage{amsthm}
\usepackage{amsmath}
\usepackage{fancyhdr}
\usepackage{tabularx}
\usepackage{hyperref}
\usepackage{ulem}
\usepackage{version}
\usepackage{thmtools}
\usepackage{qtree}
\usepackage{algpseudocode}
\usepackage{mathtools}
\usepackage{multicol}
\usepackage{enumitem}
\usepackage{ctable}
\usepackage{totcount}
\usepackage{textpos}

\XeTeXlinebreaklocale "zh"
\XeTeXlinebreakskip = 0pt plus 1pt

\setCJKmainfont[BoldFont={SourceHanSerifTC-SemiBold.otf},AutoFakeSlant]{SourceHanSerifTC-Regular.otf}
\setmonofont{CascadiaCode-Regular.otf}
\newfontfamily{\ProblemTitleMainFont}{SourceHanSerifTC-Bold.otf}
\newCJKfontfamily{\ProblemTitleCJKFont}{SourceHanSerifTC-Bold.otf}
\newcommand{\ProblemTitleFont}{\ProblemTitleMainFont\ProblemTitleCJKFont}

\pagestyle{fancy}

\lstset{
basicstyle=\footnotesize\ttfamily
}

% \raggedcolumns

\newcommand{\ProblemTitle}[2]{\noindent\Large{\ProblemTitleFont #1 (#2)}\normalsize\par}
\newcommand{\ProblemSection}[1]{\vspace{0.6cm}\par\noindent\large{\ProblemTitleFont #1}\normalsize\par}
\newcommand{\ProblemSubsection}[1]{\par\noindent{\ProblemTitleFont #1}\par}
\newcommand{\ProblemStatement}{\ProblemSection{問題敘述}}
\newcommand{\ProblemInput}{\ProblemSection{輸入說明}}
\newcommand{\ProblemOutput}{\ProblemSection{輸出說明}}
\newcommand{\ProblemConstraints}{\ProblemSection{測資限制}}

\newcommand{\ProblemSampleTitle}{\ProblemSection{範例測資}}

\newcounter{ProblemSample}
\newcommand{\ProblemSample}[2]{
    \stepcounter{ProblemSample}
    \noindent
    \begin{minipage}[t]{0.5\textwidth}
        \ProblemSubsection{範例輸入 \theProblemSample}
        \lstinputlisting{#1}
    \end{minipage}
    \begin{minipage}[t]{0.5\textwidth}
        \ProblemSubsection{範例輸出 \theProblemSample}
        \lstinputlisting{#2}
    \end{minipage}
}
\newenvironment{ProblemSampleWithNote}[2]{
    \stepcounter{ProblemSample}
    \noindent
    \begin{minipage}[t]{0.5\textwidth}
        \ProblemSubsection{範例輸入 \theProblemSample}
        \lstinputlisting{#1}
    \end{minipage}
    \begin{minipage}[t]{0.5\textwidth}
        \ProblemSubsection{範例輸出 \theProblemSample}
        \lstinputlisting{#2}
    \end{minipage}
    \vspace{-0.4cm}
    \ProblemSubsection{範例說明 \theProblemSample}
}{}

\newcommand{\ProblemSubtaskTitle}{\ProblemSection{評分說明}}
\newtotcounter{ProblemSubtask}
\newenvironment{ProblemSubtaskTable}{
    \begin{center}
        \begin{tabular}{ccl}
            \specialrule{1.3pt}{0pt}{1pt}
            子任務 & 分數 & 額外輸入限制 \\
            \specialrule{0.5pt}{1pt}{1pt}
}
{
            \specialrule{1.3pt}{1pt}{0pt}
        \end{tabular}
    \end{center}
}
\newcommand{\ProblemSubtask}[2]{ \stepcounter{ProblemSubtask} \theProblemSubtask & #1 & #2 \\ }

\setlist[enumerate]{itemsep=0pt, parsep=0pt, topsep=0pt}
\setlist[itemize]{itemsep=0pt, parsep=0pt, topsep=0pt}

\begin{document}



\renewcommand{\headrulewidth}{0pt}
\renewcommand{\baselinestretch}{1.3}
\pagenumbering{arabic}
\setlength\parindent{24pt}
\setlength\parskip{12pt}
\cfoot{\thepage}
\rhead{
	\small{SCIST S3 演算法期中考}
	
}

\ProblemTitle{D. 這是一個佇列問題}{Queue}

\ProblemStatement

你有一個字串,請寫出一支程式滿足以下操作:

\begin{enumerate}
\item 將 $x$ 個字元 $c$ 放入字串最後面
\item 將 $x$ 個字元 $c$ 放入字串最前面
\item 從字串的最後面刪除 $x$ 個字元
\item 從字串的最前面刪除 $x$ 個字元
\item 依照以下壓縮方式輸出目前的字串:將字串中連續的字母,替換為出現次數 + 字母本身,如 aaabbc 經過壓縮後會變成 3a2b1c
\end{enumerate}

\ProblemInput

輸入第一行有一個正整數 $n$ 代表操作次數

接下來的 $n$ 行,每行最一開始有一個整數 $op$ 代表操作種類,若該操作需要額外輸入參數則會依照上述的順序輸入

\ProblemOutput

對於每次操作 $5$,輸出正確的答案並換行(若字串為空,輸出空字串)


\ProblemConstraints

\begin{itemize}
    \item $1 \le n \le 1000$
    \item $1 \le x \le 10^9$
    \item $1 \le op \le 5$
    \item $c$ 為小寫字母
    \item 保證在操作 $3, 4$ 時,$x \le$ 字串長度
    \item 若需要操作 $5$,但字串為空時,則輸出空字串
\end{itemize}

\clearpage

\ProblemSampleTitle

%\begin{ProblemSampleWithNote}{ex1.in}{ex1.out}

%\end{ProblemSampleWithNote}

\ProblemSample{1in.txt}{1out.txt}


\ProblemSubtaskTitle

本題共有 5 組子任務,條件限制如下所示。

\begin{ProblemSubtaskTable}
    \ProblemSubtask{20}{不會有第 $1, 3$ 種操作,$x \le 10$}
    \ProblemSubtask{20}{不會有第 $1, 3$ 種操作}
    \ProblemSubtask{20}{不會有第 $2, 4$ 種操作,$x \le 10$}
    \ProblemSubtask{20}{不會有第 $2, 4$ 種操作}
    \ProblemSubtask{20}{無額外限制}
\end{ProblemSubtaskTable}

\end{document}