\documentclass[12pt]{article}
\usepackage{xeCJK}
\usepackage{fontspec}
\usepackage[a4paper,top=2.8cm,bottom=2.8cm,left=2.3cm,right=2.3cm]{geometry}
\usepackage{graphicx}
\usepackage{listings}
\usepackage{xcolor}
\usepackage{indentfirst}
\usepackage{tikz}
\usepackage{amssymb}
\usepackage{amsthm}
\usepackage{amsmath}
\usepackage{fancyhdr}
\usepackage{tabularx}
\usepackage{hyperref}
\usepackage{ulem}
\usepackage{version}
\usepackage{thmtools}
\usepackage{qtree}
\usepackage{algpseudocode}
\usepackage{mathtools}
\usepackage{multicol}
\usepackage{enumitem}
\usepackage{ctable}
\usepackage{totcount}
\usepackage{textpos}

\XeTeXlinebreaklocale "zh"
\XeTeXlinebreakskip = 0pt plus 1pt

\setCJKmainfont[BoldFont={SourceHanSerifTC-SemiBold.otf},AutoFakeSlant]{SourceHanSerifTC-Regular.otf}
\setmonofont{CascadiaCode-Regular.otf}
\newfontfamily{\ProblemTitleMainFont}{SourceHanSerifTC-Bold.otf}
\newCJKfontfamily{\ProblemTitleCJKFont}{SourceHanSerifTC-Bold.otf}
\newcommand{\ProblemTitleFont}{\ProblemTitleMainFont\ProblemTitleCJKFont}

\pagestyle{fancy}

\lstset{
basicstyle=\footnotesize\ttfamily
}

% \raggedcolumns

\newcommand{\ProblemTitle}[2]{\noindent\Large{\ProblemTitleFont #1 (#2)}\normalsize\par}
\newcommand{\ProblemSection}[1]{\vspace{0.6cm}\par\noindent\large{\ProblemTitleFont #1}\normalsize\par}
\newcommand{\ProblemSubsection}[1]{\par\noindent{\ProblemTitleFont #1}\par}
\newcommand{\ProblemStatement}{\ProblemSection{問題敘述}}
\newcommand{\ProblemInput}{\ProblemSection{輸入說明}}
\newcommand{\ProblemOutput}{\ProblemSection{輸出說明}}
\newcommand{\ProblemConstraints}{\ProblemSection{測資限制}}

\newcommand{\ProblemSampleTitle}{\ProblemSection{範例測資}}

\newcounter{ProblemSample}
\newcommand{\ProblemSample}[2]{
    \stepcounter{ProblemSample}
    \noindent
    \begin{minipage}[t]{0.5\textwidth}
        \ProblemSubsection{範例輸入 \theProblemSample}
        \lstinputlisting{#1}
    \end{minipage}
    \begin{minipage}[t]{0.5\textwidth}
        \ProblemSubsection{範例輸出 \theProblemSample}
        \lstinputlisting{#2}
    \end{minipage}
}
\newenvironment{ProblemSampleWithNote}[2]{
    \stepcounter{ProblemSample}
    \noindent
    \begin{minipage}[t]{0.5\textwidth}
        \ProblemSubsection{範例輸入 \theProblemSample}
        \lstinputlisting{#1}
    \end{minipage}
    \begin{minipage}[t]{0.5\textwidth}
        \ProblemSubsection{範例輸出 \theProblemSample}
        \lstinputlisting{#2}
    \end{minipage}
    \vspace{-0.4cm}
    \ProblemSubsection{範例說明 \theProblemSample}
}{}

\newcommand{\ProblemSubtaskTitle}{\ProblemSection{評分說明}}
\newtotcounter{ProblemSubtask}
\newenvironment{ProblemSubtaskTable}{
    \begin{center}
        \begin{tabular}{ccl}
            \specialrule{1.3pt}{0pt}{1pt}
            子任務 & 分數 & 額外輸入限制 \\
            \specialrule{0.5pt}{1pt}{1pt}
}
{
            \specialrule{1.3pt}{1pt}{0pt}
        \end{tabular}
    \end{center}
}
\newcommand{\ProblemSubtask}[2]{ \stepcounter{ProblemSubtask} \theProblemSubtask & #1 & #2 \\ }

\setlist[enumerate]{itemsep=0pt, parsep=0pt, topsep=0pt}
\setlist[itemize]{itemsep=0pt, parsep=0pt, topsep=0pt}

\begin{document}



\renewcommand{\headrulewidth}{0pt}
\renewcommand{\baselinestretch}{1.3}
\pagenumbering{arabic}
\setlength\parindent{24pt}
\setlength\parskip{12pt}
\cfoot{\thepage}
\rhead{
	\small{SCIST S3 演算法期中考}
	
}

\ProblemTitle{B. YTP 的記分板}{YTP Scoreboard}

\ProblemStatement

一年一度的 YTP 少年圖靈計畫快要開始報名了!

請你依照下面的規則寫出一個記分板吧!

對於一道題目,若參賽者 $x$ 尚未傳送任何一筆程式碼,那麼記分板上將顯示為 $0$

若參賽者 $x$ 在某一題獲得了 WA、TLE、MLE,記分板上該欄位的顯示數字便會 $-1$。

若參賽者 $x$ 在該題獲得了 AC(Accepted),那麼該欄位的數字就會變成正的(後面數字不變),且之後的任何 submit 都不會影響計分板上,該參賽者該題的顯示數字。

\textbf{比較特別的是,如果參賽者在第一次 Submit 就得到 AC,那記分板上會輸出} $+0$

舉例:某參賽者在 $A$ 題獲得兩次 WA,兩次 TLE,最後得到 AC,那麼對應欄位便會顯示 $+4$(注意:$0$ 以外的數字,都需要在數字前加上正負號,如 $+2$、$-3$,至於 $0$ 則是如上所述,如果是有 AC 要顯示 $+0$,沒傳送過任何 Submit 則是顯示 $0$)

請根據輸入的操作,寫一支程式模擬記分板吧!

\ProblemInput

輸入第一行有兩個正整數 $n, m, p$,代表參賽者人數、操作次數、題目數量

接下來的 $n$ 行輸入,第 $i$ 行有一個字串 $s_i$,代表記分板上第 $i$ 行的參賽者

最後 $m$ 行輸入將有幾種格式:

\begin{enumerate}
\item \{NAME\} \{PROBLEM\} \{RESULT\}:代表名字為 \{NAME\} 的參賽者,在第 \{PROBLEM\} 題傳了一次程式碼,得到的結果為 \{RESULT\}
\item view:查看程式碼
\end{enumerate}

\clearpage

\ProblemOutput

對於每次輸入為 view 的操作,輸出 $n$ 行。

每行包含 $p + 1$ 個字串,代表該行是哪位參賽者,以及該參賽者每題的結果,字串之間以空格隔開(輸出的參賽者順序要跟輸入的順序一樣)

%\clearpage

\ProblemConstraints

\begin{itemize}
\item $1 \le n, p \le 10$
\item $1 \le m \le 100$
\item $s_i$ 由小寫字母組成,長度不超過 $8$,且不會有名字叫做 view 或是有任兩個人的名字重複(意即出現 view 則一定是第二種操作)
\item 第一種操作的輸入不會有不存在的名字或題目
\item 提交程式碼得到的結果只有 AC、WA、TLE、MLE,只有 AC 被視為正確
\end{itemize}


\ProblemSampleTitle


\ProblemSample{1in.txt}{1out.txt}
\ProblemSample{2in.txt}{2out.txt}
\ProblemSample{3in.txt}{3out.txt}


\ProblemSubtaskTitle

本題共有 2 組子任務,條件限制如下所示。

\begin{ProblemSubtaskTable}
    \ProblemSubtask{50}{\{RESULT\} 不包含 AC}
    \ProblemSubtask{50}{無額外限制}
\end{ProblemSubtaskTable}

\end{document}